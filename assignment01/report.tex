%&pdflatex
%% filename: amsart-template.tex, version: 2.1
\documentclass{amsart}
\usepackage{hyperref}
\usepackage{inputenc}
\usepackage{graphicx}
\usepackage{bbm}
\usepackage{csvsimple}

\newtheorem{theorem}{Theorem}[section]
\newtheorem{lemma}[theorem]{Lemma}
\theoremstyle{definition}
\newtheorem{definition}[theorem]{Definition}
\newtheorem{example}[theorem]{Example}
\newtheorem{xca}[theorem]{Exercise}
\theoremstyle{remark}
\newtheorem{remark}[theorem]{Remark}
\numberwithin{equation}{section}
\setlength{\parindent}{0pt} % turn off auto-indent

\graphicspath{ {./} }

\begin{document}

\title{Assignment 1: [COMP550]}

\author{Joseph D. Viviano}
\address{McGill University}
\curraddr{}
\email{joseph@viviano.ca}
\thanks{}
\date{Sept 2018}

\maketitle

\section{Ambiguity}

% Phonetics
% Discourse
\subsection{Phonology}

\url{https://books.google.ca/books?id=eJUvDGXRW7AC&pg=PA92#v=onepage&q&f=false} \\

"Mr Buyer, I understand that you do not have the money to write me a check for
the retainer tonight. When would it be all \textbf{right} for you to
\textbf{write} me a check for the retainer?" \\

\textit{Right} and \textit{write} sound identical in English, but have very
different meanings. Indeed, \textit{right} after the word "all" means something
potentially different than \textit{right} on it's own. Here, \textit{right}
means "correct", and adjective, whereas \textit{write} mean "to use a hand-held
utensil  for the purposes of placing words on a medium such as paper", a verb.
For a  reader to disambiguate these words, one could look to the word before
right,  i.e., "all right", to see that right in this case means correct, where
as one  could look at the words directly after "write", i.e., "write me a
check", to see  that this word refers to the act of visual letter production. \\

\subsection{Morphology}

\url{https://twitter.com/JVM/status/1020084244490670080} \\

"We are \textbf{live} in New York City’s Brooklyn Bridge Park at the high-profile photography exhibit The Fence 2018, an exciting photo series with an animal rights message starring rescued chickens." \\

\textit{Live} by itself 'happening right now' or 'a place that I call home', and are written the same way although they sound differently when spoken. The reader must use the words immediately before it, "we are" to determine that this word "live" refers to "happening right now". \\


\subsection{Syntax}

\url{http://journals.sagepub.com/doi/abs/10.1177/1744987109358836?journalCode=jrnb} \\

"The aims of the present study were both to explore how the oldest of \textbf{old} men and women with estimated high resilience talk about experiences of becoming and being old, and to discuss the analysis of their narratives in terms of the foundational concepts of the Resilience Scale (RS)." \\

\textit{Old} in this case implies that the men are old, but it isn't clear whether this sentence applies only to old women, or all women with an estimated high resiliance. A non-ambiguous version of this sentence would be "oldest of old men and old women". In this case, the reader would have to look further to "talk about experiences of becoming and being old" to infer that all of the women being discussed here are old. \\

\subsection{Semantics}

\url{https://en.wikipedia.org/wiki/Domestication_of_the_Syrian_hamster#Capture_of_live_hamsters} \\

"The domestication of \textbf{the Syrian hamster} began in the late 1700s when naturalists cataloged the Syrian hamster, also known as Mesocricetus auratus or the golden hamster." \\

\textit{The} in this case could refer to either a single hamster of the Syrian species, or the entire species of Syrian hamsters. A non-ambiguous version of the sentence would explicity use the word 'species' i.e., "The domestication of the Syrian hamster species began in the late 1700s". To disambiguate these meanings, the reader would have to have knowledge of the most probable use of "the Syrian hamster" in English, or knowledge from the sentences before (e.g., a particular hamster was introduced). \\

\subsection{Pragmatics}

\url{https://grassrootsmotorsports.com/forum/off-topic-discussion/i-hate-squirrels/33189/page2/} \\

"\textbf{I chased} a squirrel around the yard. Up and down the yard, back and forth, back and forth and the squirrel ran up a tree. So now the car's totaled. -- Emo Phillips (a joke)." \\

\textit{I chased} without context suggests a person persuing a squirrel around the yard on foot, because that is most likely, but in fact any method of pursuit could have been employed because no particular method was defined. In this case, the reader must continue to the punchline, which makes the meaning of the first sentence concrete, i.e., the person was in an automobile. This is a cause for levity in some. \\


\section{FST for Spanish Verbal Conjugation}
\begin{table}[ht]
\centering
\resizebox{\textwidth}{!}{\begin{tabular}{l l|l|l|l|l|l}%
    \bfseries Infinitive & \bfseries 1st sg & \bfseries 2nd sg & \bfseries 3rd sg & \bfseries 1st pl & \bfseries 2nd pl & \bfseries 3rd pl % specify table head
    \csvreader[head to column names]{lex_table.csv}{}% use head of csv as column names
    {\\\hline\csvcoli&\csvcolii&\csvcoliii&\csvcoliv&\csvcolv&\csvcolvi&\csvcolvii}% specify your coloumns here
    \end{tabular}}
\end{table}
%infinitive,1-sg-reg,2-sg-reg,3-sg-reg,1-pl-reg,2-pl-reg,3-pl-reg

Please find the FST diagrams attached to the end of this document. \\

\section{Sentiment Analysis}

\subsection{Experiment Description}

To train our sentiment analysis classifier, I built a pipeline using Scikit
Learn to compare various preprocessing strategies and the Naive Bayes (NB),
Support Vector Machine with a Linear kernel (SVC), and Logistic Regression
Classifiers (LR). \\

To preprocess the corpus, ngram counts were collected for each sentence and
encoded in a one-hot manner. The data were then split into training (90\%) and
test (10\%) sets using 10-fold cross validation. All results presented are the
mean $\pm$ standard deviation across all 10 folds. This procedure was performed
independently for the NB, SVC, and LR classifiers. \\

To select the best performing hyperparameters, we used Randomized Cross
Validation to search over the possible settings. 100 randomized settings were
tried per outer fold. To evaluate these 100 randomized settings, 3 inner-fold
cross validation was performed on the training set only. The best set of the 100
best settings were then applied to the test set, ensuring no leakage between
the training and test set. The pipeline was encouraged to maximize the F1
score (macro) during hyperparameter tuning. \\

Preprocessing settings explored included removing infrequently occurring words
(less than 1-10 times in the corpus), computing unigram, unigram+bigram, and
bigram only models, and whether it is best to remove stopwords or not. \\

For the models themselves, I sampled alpha values uniformly between $0.5$ and $2$
for the NB classifier, and I sampled C values between $1\times10^{-5}$ and $100$
uniformly for the SVM and LR models. \\

Accuracy scores on the test sets, averages across all 10 folds, for all
experiments are shown below. All models performed well above chance (50\% as
there were an equal number of positive and negative examples in the dataset),
and the NB model performed best. All models show some evidence of overfitting
(nearly perfect performance on the training set with a large generalization gap)
and would benefit from stronger regularization. \\

\begin{figure}
  \caption{All classification results.}
  \centering
    \includegraphics[width=0.75\textwidth]{accs}
\end{figure}

Inner-fold cross-validation chose the following settings most often for the NB
classifier:  alpha=1.18, do not remove stop words, use a unigram+bigram model,
and to not remove infrequently occurring words. These latter three preprocessing
hyperparameters were consistent across the three classification models used,
even though these optimizations were done completely independently.

\subsection{Error Analysis}

The confusion matrix at the end of this section details the performance of the
best performing model (the Naive Bayes classifier): \\

Note that the algorithm appears to be making a similar number of errors on
positive and negative examples, so luckily our algorithm isn't too biased towards
a particular answer. To get a better sense of the kinds of errors the algorithm
made, we stored a small number of misclassified sentences across the 10 folds.
What follows are some examples: \\

\begin{itemize}
    \item{director hoffman , his writer and kline's agent should serve detention}
    \item{allen se atreve a atacar , a atacarse y nos ofrece gags que van de la sonrisa a la risa de larga duracian}
    \item{i have two words to say about reign of fire . great dragons ! }
    \item{the cast is uniformly excellent . . . but the film itself is merely mildly charming .}
    \item{mildly entertaining .}
    \item{thekids will probably stay amused at the kaleidoscope of big , colorful characters . mom and dad can catch some quality naptime along the way .}
\end{itemize}

We make a few observations. First, the model isn't picking up on what I'll call
`veiled criticisms'. For example, saying that the filmmakers should
\textit{serve detention} or that the film is \textit{mildly entertaining} is
an indirect criticism in the first case, and actually a positively-valanced
sentiment in the second case. A better algorithm would need a fuller understanding
of language beyond bigrams to understand that `mildly entertaining' might actually
not be an endorsement in the context of a film review. Second, at least one review
in the corpus is not in English. Any other foreign language reviews we would expect
to be predicted at chance level (50\%). Third, some reviews use a lot of positive
words in isolation as a form of either irony or politeness (e.g.,
\textit{mom and dad [can nap]}, \textit{great dragons}, \textit{mildly charming}).
These sentiments imply a negative review to the reader (`are the dragons the
only good part of this film?'), but a bigram model might read `great' or `great
dragons' and reasonably score this as a positive review. Fourth, it appears that
some words were not separated properly during preprocessing (i.e., \textit{thekids}
and \textit{naptime}). These would be treated as new words in the corpus, which isn't
appropriate, as these errors aren't likely to occur often. \\

I suggest: better data cleaning to remove foreign language reviews, a more sophisticated
preprocessing pipeline that will catch joined words and other similar errors, and a more
sophisticated model that understand language in context beyond what a bigram model can
encode.

\begin{table}[]
\begin{tabular}{lll}
                  & \textbf{Positive} & \textbf{Negative} \\
\textbf{Positive} & 427.0\pm10.35     & 106.1\pm10.28   \\
\textbf{Negative} & 122.5\pm8.88      & 410.6\pm8.85
\end{tabular}
\end{table}

\end{document}
